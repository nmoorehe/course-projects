\documentclass{amsart}

\begin{document} 

\title{Bifurcation of the Logistics Equation}
\section {introduction}
Owen Angleton, James Garrett, Kelli McWhorter, Perounsack Moon,\\
Padmavathi Sarvepalli were assigned to investigate period doubling.\\
For our project we investigated the onset of chaos in the logistics equation.\\
First we reviewed the section of the textbook that included the idea of period\\
doubling and bifurcation.\\
\\
We wrote the code in the code section to run in MATLAB or FreeMAT and produced an output\\

\section {Code}
Due to MATLAB and LATEX sharing comment codes, the code is attached.\\

\section{Results}
The code ran 20 iterations of the equation and produced graphs of the general\\
bifurcation pattern for values of R between 2 and 4, and of the periods for \\
some randomly selected values of R.  The period doubling is evident in the R \\
values near the bifurcation zones, while chaos is evident in the black areas \\
of the bifurcation graph.\\

The graphics are attached.\\

Graph 1 is the bifurcation chart.\\


Graph 2 is the period chart for R=2.1.  Note the stable constant value.\\

Graph 3 is the period chart for R=2.8. Note an alternation between two 
decaying peaks tending toward a central constant value.  \\

Graph 4 is the period chart for R=3.0.  This is near the onset of a stable doubled\\
period, but it still is trying to decay toward a stable value.\\

Graph 5 is the period chart for R=3.21.  Here the doubled period is stable.\\

Graph 6 is the period chart for R=3.4.  Here we are seeing two period doubles \\
superimposed upon one another for a total of 4 periods.  This corresponds to \\
the second bifurcation zone on the bifurcation chart.\\

Graph 7 is the period chart for R=3.45.  Here the second doubling is even more\\
obvious as this is more firmly into the second stable hole in the bifurcation\\
chart\\

Graph 8 is the period chart for R=3.5.  Here we are seeing more clearly the\\
trend seen in Graph 6.  The peaks for the new periods are more profound.\\


Graph 9 is the period chart for R=3.81.  Here we are seeing a chaotic pattern\\
The peaks become irregular and no set pattern is discernible\\


Graph 10 is the period chart for R=3.85.  In this graph, the pattern has a \\
distinctly different shape and its period is 3 rather than multiples of 2 as\\
in the previous cases.  This value corresponds to the last gap in the bifurcation\\
chart, which shows 3 stable values.\\


Graph 11 is the period chart for R=3.95. Chaos is evident.\\




\end {document}